\documentclass[12pt, oneside]{article}  

%\documentclass[11pt, twocolumn,left=1cm,right=1cm]{article} % Default font size is 12pt, it can be changed here
%Saved into templates Feb 22th 2017. Served me well through college!
\usepackage{amsmath, amssymb, mathtools, physics, amsthm, braket, tensor}
\usepackage[xetex]{graphicx}
\usepackage{fontspec,xunicode}
\defaultfontfeatures{Mapping=tex-text,Scale=MatchLowercase}
%\setmainfont[Scale=1.0]{Times}
\setmonofont{Times}
\usepackage{lipsum, dsfont, fancyhdr, graphicx}

\usepackage[margin=1in]{geometry}
%\geometry{a4paper} % Set the page size to be A4 as opposed to the default US Letter

\setlength{\headheight}{15pt} 

\linespread{1} % Line spacing

\renewcommand{\thispagestyle}[1]{} % This does nothing, just defines what the page style is.

\usepackage[labelformat=empty]{caption}

\newcommand{\fig}[2]{ 
\begin{figure}[bth] 
\centering
\includegraphics[width = 10cm]{#1} 
\caption {#2} 
\end{figure}}

\newcommand{\eqn}[1]{\begin{equation}#1 \end{equation}}
\renewcommand{\thesubsection}{\thesection.\alph{subsection}}
\renewcommand{\dag}[1]{#1^{\dagger}}
\newcommand{\dagg}[1]{#1^{\dagger 2}}
\newcommand{\daggg}[1]{#1^{\dagger 3}}
\newcommand{\infint}{\int_{-\infty}^{\infty}}
\newcommand{\algn}[1]{\begin{align}#1 \end{align}}
\newcommand{\e}{\epsilon}
\newcommand*{\QED}{\hfill\ensuremath{\square}}
\usepackage{tensor}

\numberwithin{equation}{section}


\begin{document}


\pagestyle{fancy} % Calling the package
\lhead{\textit{Steven B. Torrisi}} % Left side of header
%\rhead{\textit{Set 1}} % Right side of header % Left side of header
\title{ Course Title }
\date{Due Date}
\author{Steven B. Torrisi} 

\maketitle

\section{Problem}

\subsection{Part A}

$$h_{A_{t} O_{t}}^{A_{f} O_{f}} \rightarrow A_f=\text{Atom From} \  A_t=\text{Atom to} $$ 
$$O_f=\text{Orbital From} \ O_t = \text{Orbital To}$$


$$ H = \mqty( 
h_{1,1}^{1,1}      & h^{1,1}_{1,2} &h^{1,1}_{1,3} & \dots & h^{1,1}_{2,1} & h^{1,1}_{2,2} & \dotsi  \\
h_{1,1}^{1,2}      & h^{1,2}_{1,2} &h^{1,2}_{1,3} & \dotsi &h^{1,2}_{2,1} & h^{1,2}_{2,2}  & \dotsi  \\
\vdots                 & \vdots            & \vdots           & \ddots & \vdots          & \vdots           &         \\
h^{1,11}_{1,1}     & h^{1,11}_{1,2}& \dotsi           &	      & \ddots         &		          &      	\\
h^{2,1}_{1,1}      & h^{2,1}_{1,2} & \dotsi             &         &                      & \ddots           &          \\
\vdots                  & \vdots            & \vdots             &        &                     &                      &    ) 
$$

\end{document}

